\documentclass[12pt]{article}

\usepackage[utf8]{inputenc}
\usepackage{amsmath, amssymb}
\usepackage{mathtools}
\usepackage{geometry}
\usepackage{physics}
\usepackage{color}

\geometry{margin=1in}

\title{Greater number}
\author{Álvaro Fernández Barrero}
\date{16/10/2025}

\begin{document}
\maketitle

\[
     1,4 KB = (1,4 \cdot 8 \cdot 1024) bits = 11468,8 \; bits
\]
\[
    \therefore 1,4 \; KB < 583254 \; bits
\]

\[
    1,2 TB = (1,2 \cdot 1024^2)KB = 1258291,2 \; KB
\]
\[
    \therefore 1,2 \; TB > 1200000 \; KB
\]

\[
    328921 Bytes = \frac{318921}{1024^2}MB = 0,31368351 \; MB
\]
\[
    \therefore 328921 \; Bytes > 0,3 \; MB
\]

\[
    20365987 \; bits = \frac{20365987}{8 \cdot 1024}KB = 40,151489258 \; KB
\]
\[
    \therefore 0365987 \; bits < 2400 \; KB
\]

\[
   67200 \; bits = \frac{67200}{8} Bytes =  8400 \; Bytes
\] 
\[
    \therefore 67200 \; bits < 8400 \; Bytes
\]

\[
    8400 \; Bytes = \frac{8400}{1024}KB = 8,203125KB
\]
\[
    \therefore  8400 \; Bytes > 8,1 KB
\]

\[
    0,06 TB = (0,06 \cdot 1024) GB = 61,44 \; HB
\]
\[
    \therefore 64 GB > 0,06 TB
\]

\end{document}
