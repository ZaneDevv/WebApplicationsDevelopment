\documentclass[12pt]{exam}

\usepackage[utf8]{inputenc}
\usepackage{amsmath, amssymb}
\usepackage{mathtools}
\usepackage{geometry}
\usepackage{physics}
\usepackage{color}

\geometry{margin=1in}

\title{Candidate keys}
\author{Álvaro Fernández Barrero}
\date{November 2025}

\begin{document}
\maketitle

\begin{questions}
\question
Given the table $R(X, Y, Z, W, T)$ and the following functionals relations $D_F=\{X \to W, Y \to T, XY \to Z\}$. Write the candidate keys.

Since $X$ determinates $W$, $Y$ determinates $T$ and the mix of $XY$ determinates $Z$, only by using $X$ and $Y$ we can determinate every single one. Thus, $XY$ must be a the \emph{candidate key}.

\;

\question
Now, given the table $R(A, B, C, D, E, F)$ and the functionals $D_F=\{AB \to C, D \to E, C \to F, F \to DBA \}$. Write the candidate keys.

The candidate keys here are $C$, $AB$ and $F$.

\;

\question
Make up your own functional relations on the previous table $R(A, B, C, D, E, F)$ in such a way that there are exactly three candidate keys.

\[
    D_F = \{ A \to B, C \to E, D \to F \}
\]

Hence, the \emph{candidative keys} are $A$, $AB$ and $ACD$.


\end{questions}
\end{document}
